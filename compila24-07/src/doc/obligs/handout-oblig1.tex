\documentclass[10pt,freeform]{handout}[2014/08/13]

%\dozent{\mbox{}}
\def\myauthor{\href{https://martinsteffen.github.io}{Martin Steffen}}
\def\mytitle{INF 5110: Compiler construction}

\coursename{INF 5110: Compiler construction}
\semester{Spring 2024}
\dozent{Martin Steffen}
\ifigroup{Reliable systems}


%%% Local Variables: 
%%% mode: latex
%%% TeX-master: t
%%% End: 

%
%\newcommand{\deadlineone}   {Friday, 15. 03. 2019}
%\newcommand{\deadlineone}   {Friday, 06. 03. 2020}
%\newcommand{\deadlineone}   {Friday, 12. 03. 2021}
%% easter 17 April 2022
% \newcommand{\deadlineone}   {Friday, 11. 03. 2022}
% \newcommand{\deadlineone}   {Friday, 17. 03. 2023}
\newcommand{\deadlineone}   {Friday, 22. 03. 2024}


%\newcommand{\deadlinetwo}   {--11. May  2018--}
%\newcommand{\deadlinetwo}   {--12. May  2019--}
%\newcommand{\deadlinetwo}    {12. May 2020}
%\newcommand{\deadlinetwo}    {14. May 2021}
%\newcommand{\deadlinetwo}    {09. May 2022}
\newcommand{\deadlinetwo}    {Friday, 12. May 2023}



%%% Local Variables: 
%%% mode: latex
%%% TeX-master: "handout-oblig2.tex"
%%% End: 



\pgfdeclareimage[height=2.4cm,interpolate=true]{uio}{logos/uiologo}%% relative
%\usepackage[german]{babel} 
%\usepackage{german}

\handouttitle{Oblig 1}
\handoutnumber{1}
%\issuedate{12.\ 02.\ 2019}
%\issuedate{03.\ 02.\ 2020}
%\issuedate{09.\ 02.\ 2021}
%\issuedate{8.\ 2.\ 2022}
%\issuedate{10.\ 2.\ 2023}
\issuedate{12.\ 2.\ 2024}
\topic{---}




\begin{document}
\thispagestyle{empty}

\section{Official grading guidelines}
\label{sec:official-info}




\hrulefill{}

The \textbf{deadline/frist}  for the oblig is

\begin{quote}
  \textbf{\deadlineone, 23:59} %% 31 March is Ostern
\end{quote}

Requirements:


\begin{itemize}
\item the test program parses 100\% ok (see below)
\item the printed AST reflects the correctly parsed structure (especially,
  the correct associativity and precedence). The exact same formatting as
  in the provided illustration is not required; reasonable deviations are
  fine.
\item follow the instructions under ``What and how to  hand-in''
\item the solution needs to compile and run \emph{at the UiO
    pool}.\footnote{Statements like ``but on my laptop it worked, I can show
    you'' don't cut it \ldots.} Test it!
\end{itemize}

\hrulefill




\section{What and how to hand in}
\label{sec:what-how}


\subsection{Git}
\label{sec:git}



We use
\begin{quote}
  \textbf{git}   
\end{quote}
to hand in the obligs and the two deliverables are just two ``milestones''
in your project. We encourage working in groups, and git allows to share
things.

I assume that most are in principle familiar with \texttt{git}, if not, ask
me.  We use the UiO-internal git server\footnote{Actually, it's the UiO
  enterprise github server. Github is just one particular ``web-interface''
  on top of git which does the real work; there are also alternatives to
  github (and git) but for uniformity, we stick to the UiO github server.}

\begin{quote}
  \url{github.uio.no}
\end{quote}


In order to hand-in via git, each group has to do the following steps. See
also the top-level readme at
\url{https://github.uio.no/compilerconstruction-inf5110/compila/} (resp. at
the corresponding place in your clone).

\begin{enumerate}
\item everyone: if not already: create yourself an account at
  \url{github.uio.no}. Your UiO login allows you to do that.
\item send me an email with your name (and if you have, with the name of
  your collaborator), so I can assign a group number.
\item each group will get one \emph{repository}. For instance, group
  \textbf{01} get the repository \textbf{compila24-01} etc.  I will
  everyone to the corresponding repos as collaborator. Each repos contains
  the same starting point, of course.
%\item per group: \emph{create} a new \emph{project} at the github repos.
%  If your are in group number $n$,\footnote{The project names need to be
%    different so that, for correction, I can distinguish them by their
%    ``name'' to avoid that all projects are called ``compila''.}
%  \begin{quote}
%    call your project \texttt{compila<n>}
%  \end{quote}
%  The ``parentheses'' \texttt{<} and \texttt{>} are \emph{not} part of the
% name! If the group contains more than one person, the creator has to add
%  the partner as \emph{collaborator} to the project.\footnote{It's not that
%    the one creating the project ``owns'' it, the collaborators are
%    \emph{peers} (as in peer-to-peer). As you know, in git, everyone works
%    on a cloned copy of the respository, so at the end of the project,
%    everone can keep the own copy if wanted, not only the one who created
%    it.}
%\item If your project is ``private'': add me as collaborator (login
%  \texttt{msteffen}). I don't need to contibute as collaborator, but I need
%  access.
% \item Send me an \emph{email} with the link, mentioning the names (and
%   login) of the members of the group as confirmation. That needs to be done
%   \emph{before the first oblig's deadline}. The names of the member of the
%   group, i.e., the authors of the project, should also feature
%   \emph{prominently} on the top-level of the repository, as in the
%   top-level \texttt{Readme}-file.
\end{enumerate}


\subsection{What to include into a solution}
\label{sec:what-include-into}


\begin{itemize}
\item A top-level \emph{Readme-file} containing
  \begin{itemize}
  \item names and emails of the authors
  \item easy instructions how to build the compiler and how to run it.
  \end{itemize}
  The top-level Readme should describe in a concise manner instructions for
  building, installation, running, and testing your compiler in a manner
  useful for an interested \emph{user}. The target audience is a
  master-level computer science student or someone how is not afraid to
  git-clone a repository (or download a jar-file or similar) and following
  a few clearly installation steps. It should be doable ``stand-alone'',
  i.e., without loading some files into Eclipse or similar \emph{developer}
  tools (developer $\not=$ user). The user is not expected to figure out
  himself or herself how to install, run, and test it. The user is also
  \emph{not} expected to be a student of this compiler course and familiar
  with the specification of the compila-language and the
  oblig-documentation.

  It might be nice to use some \emph{markdown} format (like a
  \texttt{Readme.md}). It's some form of poor-man's markup, and is
  typically rendered nicely by browsers. The current top-level
  Compila-Readme is not in the \texttt{md}-format, but in
  \texttt{org}-format (\texttt{Readme.org}) (which is a similar, but a more
  expressive markdown format and also rendered nicely by github, and which
  personally I prefer for myself).
  
\item The information so far is for an ``interested user'' which may be an
  outsider, maybe like a random surfer on the internet who wants to try it
  out. Besides that, \emph{additional} information should be provided,
  which is not relevant for an outsider, but contains ``insider
  information'' relevant for the course's mandatory assignment and the
  technical realization of the solution. Thus, that information might
  perhaps better not be kept in the top level readme but at a different,
  less prominent place or file.
  \begin{itemize}
  \item test-output for running the compiler on \texttt{compila.cmp} as
    input
  \item of course, the \emph{code} needed to run your package. That
    includes
    \begin{itemize}
    \item \textsl{JFlex}-code for the scanner
    \item \textsl{CUP}-cpde for the 2 variants of the syntax
    \item the Java-classes for the syntax-tree
    \item the build-script \texttt{build.xml}\footnote{Alternatively, you
        can use a makefile if you prefer that. In that case, remove the
        \texttt{build.xml}-file.}
    \end{itemize}
  \end{itemize}
\end{itemize}



\subsection{Delivery in a nutshell}
\label{sec:delivery-nutshell}


Ideal would be an 
\begin{center}
  \textbf{out-of-the-box}
\end{center}
behavior. Assuming that the ``outside user'' has the required software in
place (like \texttt{JDK} and \emph{ant}, or something rather common, and
also is familiar with git). Then it works out-of-the-box, by following
simple steps like for example


\begin{verbatim}
   git clone <repos-url>
   cd compila24-<n>
   ant
   ant build
   ant run
\end{verbatim}
and those steps (whatever they are) should be mentioned clearly in your
Readme. The steps mentioned here correspond to those suggested when using
the provided starting point, which corresponds to
\begin{center}
  \url{https://github.uio.no/compilerconstruction-inf5110/compila}.\footnote{Don't
    clone this one for doing the oblig, clone the repos corresponding to
    your group. If you clone it, it's the same information, but you have no
    write access to the repos.}
\end{center}
Your solution may involve different steps (maybe \texttt{ant ast} for
producing the require AST), and may require adaptation of \texttt{build.xml}
(if you use ant). But you should \textbf{keep the spirit} of
``out-of-the-box'' and ``kiss'' (keep it simple, stupid.). Remember: the
user is always stupid...





\section{Purpose and goal}
\label{sec:x}

The goal of the task is to gather practical experience of the following
tools and techniques.
\begin{itemize}
\item use scanner/lexer and parser tools. In this case \texttt{JFLex} and
  \texttt{CUP}.
\item rewrite and massage a grammar given in one form into another one so
  that it's accepted by the tools. In our case, the language is given in
  some EBNF, which has to be adequately rewritten so that it can be fit
  into the lexer and parser tools
\item handle associativity and precedence of syntactic constructs in two
  possible ways
  \begin{itemize}
  \item formulate a (unambiguous) grammar that embodies the correct
    precendences and associativities
  \item work with an ambiguous grammar, but instruct the parser tool (like
    CUP) to result in an appropriate parser.
  \end{itemize}
\item design and implement an suitable AST data structure.  Use the parser
  to output your AST (in case of a successful parse).
\item do a ``\emph{pretty printer}'' in the following sense: implement some
  functionality that \emph{outputs} and AST in a ``useful'' manner. In
  particular, the parenthetic tree structure must be visible from the
  output (i.e., one can see whether the associativity and the precedence is
  correctly implemented). 
\end{itemize}


\section{Tools}
%\label{sec:tools}

The platform is \emph{Java}, together with the auxiliary tools

\begin{itemize}
\item \textsl{JFlex} (scanner generator in the (f)lex family)
\item \textsl{CUP} (parser generator in the yacc family)
\item \textsl{ant} (a kind of ``make'' tool specialized for Java)
\end{itemize}


The tool ant is available at the RHEL pool at IFI, for other platforms I
don't know, but it's freely available. \textsl{JFlex} and \textsl{CUP} are
provided as part of the repos.

If, for some reason, you plan to deviate from the suggested tools, you
\begin{enumerate}
\item \textbf{MUST} discuss that first with the lecturer.
\item it \textbf{must} be a platform which is freely available at the
  university RHEL pool resp. is platform independent. Proprietary tools or
  tools I don't have easy\footnote{I \emph{mean} easy and in the sense that
    it does not cost money or much time to install the required environment
    or to figure out how it all hangs together. \emph{Not} ``easy'' as in
    ``it's really not so hard after you read some manuals and played around
    with it for a while and with the help from the fine folks on
    stack-exchange'' \ldots} access to cannot be used. If using Java, it
  must compile and run without support of specific development environments
  or ``frameworks'' besides the ones mentioned (\textsl{JFlex},
  \textsl{CUP}, \textsl{ant}).
\end{enumerate}


\section{First steps (``oblig 0'')}
\label{sec:first-steps-oblig}

Apart from getting yourself a group number, git access etc. as described,
there are things you really should do afterwards:

\begin{center}
  \textbf{Checking that on your platform, all is in place}
\end{center}

The provided readme describes the easy steps (not repeated here). It's
about to make sure that your Java set-up works, that ant is running,
etc. If the tests fail, \textbf{contact me} to \textbf{get that out of the
  way} before the oblig work starts. The starting point provides those
tests to avoid that later and shorter before the deadline, one figures out
``something's not working on my machine''. I tested it on the university
machines, I tested it on my laptop, but I cannot test it on other people's
machines.  Normally it's not a problem, the stuff is supposed to be
platform independent, but ``normally ok'' means, sometimes, maybe only
rarely, not ok, and sometimes for weird reasons.

To avoid running into frustrating stupid things later is \textbf{only one
  purpose} to run the provided the tests and make sure all is in order. The
\textbf{other purpose} is: the tests run a lexer and a parser!  It's of
course not the compila language, that's your task, but it shows in two tiny
examples, how the tools operate together, and how a lex file looks and how
a cup-file looks etc. And for that:

\begin{center}
  \textbf{Do that checks immediately} even if you don't plan to start on
  the content your oblig right away.
\end{center}
It's just a test that all is in place.

Talking about how to tackle practical aspects of the oblig(s) and
organizing the task. You may read the advice I typed up last year about that

\begin{center}
  \url{https://martinsteffen.github.io/compilerconstruction/howtowriteaparser/}
\end{center}

% Sometimes, groups or individuals work on the conceptual part of the
% compiler (like writing the lex-file or the grammar) which is of course very
% important. But postponing to actually \textbf{check} whether it ``runs'' or
% ``fit together'', which may result in hectic before the deadline.




\section{Task more specifically: Syntax check and parsing}
\label{sec:task-more-spec}


The overall task is to 

\begin{quote}
  implement a parser for the \textsl{Compila 24} language.
\end{quote}
The language specification is given in a separate document. Oblig 1 is
concerned with checking \emph{syntactic correctness}, which means, not all
of the language specification is relevant right now: semantic correctness,
type checking etc. will become relevant only later for the second oblig.


\subsection{Syntax tree}
\label{sec:syntax-tree}

The result of a successful parse is an \emph{abstract syntax tree}. That
data structure needs to be appropriately ``designed''. In a Java
implementation, that involves the definition of appropriately chosen
classes arranged in some class hierarchy. Make also use if \emph{abstract
  classes}. In the lecture, there had been some ``design guidelines'' that
may be helpful. Carefully chosen names for classes will help in a
conceptually clear implementation. A definitely \emph{non-recommended} way
is to have one single class \texttt{Node} lumping together all kinds of
nodes and syntactic categories in the syntax tree.









\subsection{Print out of the AST}
\label{sec:print-out-ast}


The AST should be ``printed''. The easiest and recommended form of printout
is in \emph{prefix form}. Under \texttt{material/sample-compila-ast}, there
is an example compila input file and a corresponding file containing a
possible output. The two files are called

\begin{itemize}
\item \texttt{complexaddition.cmp}
\item \texttt{complexaddition.ast}
\end{itemize}


Note: the two files are meant as \emph{inspiration.} Each year the syntax
of \textsl{compila} slightly changes (wrt. keywords, associativity etc). So
the syntax is \textbf{not 100\%} in accordance with the 2024 version (but
pretty comparable).

The one that \textbf{should actually be used} for this task (AST-printing)
is the following:

\begin{quote}
  \texttt{./src/tests/fullprograms/complexaddition.cmp}  
\end{quote}

and this one should be consistent with this year's grammar (fingers
crossed).



It's allowed (but not necessary) to print it in other forms than the prefix
form used in the inspiration. But the output must indicate the AST in
readable form (``readable'' as in human-readable that is \ldots). Note, the
task is not that the output is a syntactically correct \textsl{compila}
program again (that might be a so-called formatting tool or a
pretty-printer), we just need a way to look at the syntax tree, which comes
in handy for debugging,


%\bibliographystyle{apalike}
%{\small
% \bibliography{string,semantics,crossref}
%% \bibliography{extracted}
% }

\subsection{Two grammars}
\label{sec:two-grammars}


As mentioned shortly, the task requires 2 grammars, representing 2 ways
dealing with precendence and associativity.

\begin{enumerate}
\item an \emph{unambiguous} grammar resolving precedence and associativity
  by ``baking it in'' directly into the grammar. The grammar is in plain
  BNF (in the form required by the tools)
\item the second grammar is ambiguous and relies on \emph{CUP} to resolve
  the associativity and precendence. This second grammar will probably look
  nicer and will be shorter. It's therefore probably best to take that one
  as \emph{default} (for instance for oblig 2).
\end{enumerate}

\subsubsection*{Comparison and discussion}
\label{sec:comparison}

Investigate and characterize \emph{conflicts} of the \emph{original}
grammar. How many states do the 2 generated CUP grammars have? That
requires a look into the CUP-generated code. Discuss also whether the
choice of the two grammars influences the generation of the AST: is one of
the two approaches easier to work with when it comes to generate an AST
(resp. your chosen AST data structure.


\textbf{Note:} It's not required to provide code to build \emph{two}
versions of AST-generation, one is enough. In other words, for one of the
two grammars, you don't need ``action code'' in the grammar to produce an
AST, plain \emph{checking} is sufficient.

\subsection{Lexical analysis}
\label{sec:lexical-analysis}

As mentioned, \textsl{JFLex} is the tool of choice for lexical analysis. It
delivers a token to the parser via the method \texttt{next\_token()}.  

As far the the ``theoretical'' task concerning \textsl{compila 24} is
concerned, the lexer is responsible for ignoring comment, white-space etc,
find keywords and the like.

Besides that, one has to make the parser and the lexer ``work together''
hand in hand. Information about that can be found in the corresponding
manual. There should also be examples for inspiration. A crucial ingredient
is the interface \texttt{java\_cup.runtime.Scanner} which needs to be
implemented by the actual scanner. The scanner will hand over tokens of the
type \texttt{Symbol} and one can use \texttt{Symbol.value} to pass ``text''
or other objects from the lexer to the parser.



\subsection{Error handling}
\label{sec:error-handling}

Error handling can be done simple: When hitting an error, parsing should
stop (as opposed to try to continue and give back an avalanche of
subsequent errors). Some meaningful error message (at least wrt. which
syntactic class caused the error) would be welcome, as opposed to a plain
``sorry, bad program''. It's not required to give back line numbers
referring to the original source code or positions in the original
file. In practice that's definitely useful (and not very hard either), but
not required for the oblig.




\subsection{Tests}
\label{sec:tests}

For testing, there is a bunch of files under

\begin{quote}
  \texttt{./src/tests}
\end{quote}

They are supposed to contain syntactically correct programs for this year's
version, with exception of the ones under
\begin{quote}
  \texttt{./src/tests/errors}
\end{quote}
Later for \textbf{oblig 2} there will be more tests, especially for testing
type checking and code generation. 


For \textbf{oblig 1}, you are requested to generate an ast for
\begin{quote}
  \texttt{./src/tests/fullprograms/complexadditions.cmp}  
\end{quote}
as part of the oblig, as mentioned, but you may of course use the other
test program as well to see how robust your implementation is. 

% Especially erroneous programs later will become relevant for oblig2,
% when we do type checking; there, some syntactically ok programs should
% be flagged as erroneous by the type checker.







\section{Resources}
\label{sec:resources}

The web-page (the git-one) will contain also links to \textsl{JFlex}
\textsl{CUP} and corresponding manuals.






\end{document}

Model Checking Cache Coherence Protocols for Distributed File Systems

%%% Local Variables: 
%%% mode: latex
%%% TeX-master: t
%%% End: 
